% Options for packages loaded elsewhere
\PassOptionsToPackage{unicode}{hyperref}
\PassOptionsToPackage{hyphens}{url}
%
\documentclass[
]{article}
\usepackage{lmodern}
\usepackage{amssymb,amsmath}
\usepackage{ifxetex,ifluatex}
\ifnum 0\ifxetex 1\fi\ifluatex 1\fi=0 % if pdftex
  \usepackage[T1]{fontenc}
  \usepackage[utf8]{inputenc}
  \usepackage{textcomp} % provide euro and other symbols
\else % if luatex or xetex
  \usepackage{unicode-math}
  \defaultfontfeatures{Scale=MatchLowercase}
  \defaultfontfeatures[\rmfamily]{Ligatures=TeX,Scale=1}
\fi
% Use upquote if available, for straight quotes in verbatim environments
\IfFileExists{upquote.sty}{\usepackage{upquote}}{}
\IfFileExists{microtype.sty}{% use microtype if available
  \usepackage[]{microtype}
  \UseMicrotypeSet[protrusion]{basicmath} % disable protrusion for tt fonts
}{}
\makeatletter
\@ifundefined{KOMAClassName}{% if non-KOMA class
  \IfFileExists{parskip.sty}{%
    \usepackage{parskip}
  }{% else
    \setlength{\parindent}{0pt}
    \setlength{\parskip}{6pt plus 2pt minus 1pt}}
}{% if KOMA class
  \KOMAoptions{parskip=half}}
\makeatother
\usepackage{xcolor}
\IfFileExists{xurl.sty}{\usepackage{xurl}}{} % add URL line breaks if available
\IfFileExists{bookmark.sty}{\usepackage{bookmark}}{\usepackage{hyperref}}
\hypersetup{
  pdftitle={girişimci havuzunun gittikçe verimsizleşmesi hakkında},
  pdfauthor={Orhan Aktaş},
  hidelinks,
  pdfcreator={LaTeX via pandoc}}
\urlstyle{same} % disable monospaced font for URLs
\usepackage[margin=1in]{geometry}
\usepackage{graphicx,grffile}
\makeatletter
\def\maxwidth{\ifdim\Gin@nat@width>\linewidth\linewidth\else\Gin@nat@width\fi}
\def\maxheight{\ifdim\Gin@nat@height>\textheight\textheight\else\Gin@nat@height\fi}
\makeatother
% Scale images if necessary, so that they will not overflow the page
% margins by default, and it is still possible to overwrite the defaults
% using explicit options in \includegraphics[width, height, ...]{}
\setkeys{Gin}{width=\maxwidth,height=\maxheight,keepaspectratio}
% Set default figure placement to htbp
\makeatletter
\def\fps@figure{htbp}
\makeatother
\setlength{\emergencystretch}{3em} % prevent overfull lines
\providecommand{\tightlist}{%
  \setlength{\itemsep}{0pt}\setlength{\parskip}{0pt}}
\setcounter{secnumdepth}{-\maxdimen} % remove section numbering

\title{girişimci havuzunun gittikçe verimsizleşmesi hakkında}
\author{Orhan Aktaş}
\date{2020-07-02}

\begin{document}
\maketitle

Ekosistem verimsizleşiyor. Peki neden? Bunu anlamak için önce şu anda
işleyen mekanizmaya bakmamız, büyük resmi görmemiz gerekiyor. Daha sonra
bu büyük resim içinde mekanizmanın neden verimsize doğru gittiğine
bakılması lazım. Hadi başlayalım.

Şu andaki girişim havuzunu geçen seneki girişimci havuzu üzerinden
formülize edelim. 2019 yılındaki girişimci havuzundan ayrılanlar oldu bu
aradaki dönemde. Diğer yandan yeni umutlarla bu havuza dahil olanlar da
oldu. O zaman basit olarak bu seneki girişimci havuzunu şu şekilde
hesaplayabiliriz.

\(Girişimciler_{2020} = Girişimciler_{2019}-Gidenler+Gelenler\)

Bu denklemde gelenler hakkında çok konuşulurken, maalesef aslında esas
önemli olan gidenlerden çok az bahsediliyor. Aslında ekosistemi büyük
bir şirket olarak düşünürsek şu anda bütün stratejisi daha fazla istekli
insanı girişimciliğe çekmek gibi görünüyor. Girişimci hikayeleri, pembe
hayaller, yatırımların giderek arttığı, paranın çok bol olduğu, kendi
işini yapmanın, sıfırdan bir şey ortaya koymanın cazibesi\ldots{} Ben bu
yazıda sadece gidenler hakkında da bir şeyler söylemiyor, çok daha ileri
giderek, gidenleri sistemde tutmanın Türkiye'deki girişimciliği
geliştirmek için yapılabilecek en iyi strateji olduğunu anlatmaya
çalışıyorum.

Peki kimdir bu gidenler? Bu kişiler önceki senelerde gelenlerin bir alt
kümesi doğal olarak. Onlar da bir zamanlar yeni gelenler gibiydiler.
Yıllar içinde bir şeyler denediler ve farklı nedenlerden ötürü
ayrılıyorlar. Bu aynı zamandan şu doğal sonucu da getiriyor: Gidenlerin
gelenlerin bir kaç sene sonraki, deneyim kazanmış halleri aslında. Bu o
kadar önemli bir nokta ki, bence tüm stratejinin üzerine kurulması
mantıksız olmaz. Mevcut durumda tecrübelileri gönderip, acemileri
alıyoruz. Halbuki bu kadar tecrübelenmiş kişileri tutmamız gerekmez mi?
Normalde girişimcilikle ilk tanışıyorsanız öğrenmeniz gereken en az 5-6
tane sağlam alan var. Şirket kurma, iş geliştirme, muhasebe, vergiler,
finansman yönetimi, ortak bulma gibi konuları öğrenmek, tecrübe etmek ve
tabi kavramak deneme yanılma ile oluyor ve bu da oldukça fazla zaman
alıyor. Tam bunları öğrenmişken girişimcilerin finansmanları kalmıyor ve
ekosistemden ayrılıyorlar. Bu grup içinde bu işin kendileri için uygun
olmadığını görüp ayrılanlar da var elbette ancak girişimciliğim
süresince benim gözlemim büyük çoğunluğunun finansman yetersizliğinden
ayrıldığı, daha doğrusu ayrılmak zorunda kaldığı yönünde. Bir çok şeyi
en etkili öğrenme yolu olan deneme yanılma ile öğrenmiş ve daha verimli
olmaya başlama arifesindeki girişimcileri içeride tutmamız lazım.

\hypertarget{eleux15ftiriler}{%
\subsection{Eleştiriler}\label{eleux15ftiriler}}

\hypertarget{doux11fal-seleksiyon}{%
\subsubsection{1) Doğal seleksiyon}\label{doux11fal-seleksiyon}}

Bu fikrin en temel karşı argümanı elbette burada doğal seleksiyon
eleştirisidir. Yani bu işin doğasının böyle olduğu ve uygun olanların
kaldığı, uygun olmayanların yapamayıp ayrıldığını ifade eden bu
yaklaşımın bu konuda çok doğru olduğunu düşünmüyorum. Çünkü zaten işin
özünde bir şans faktörü var. Hangi startup'ın başarılı olacağı dünyanın
hiç bir VC'si tarafından bile bilinemezken aslında kalanlar şansı yaver
gidenler ya da daha fazla finansmanı olanlar oluyor. Bu iki konuyu da
anlamsız şekilde yücelten kişiler var ve ben aşağıdaki nedenlerle bu
kişileri anlamakta güçlük çekiyorum.

\hypertarget{a-ux15fans-konusunda}{%
\subsubsection{a) Şans konusunda}\label{a-ux15fans-konusunda}}

genişlet mistik bakış açılarıyla ``\emph{girişimci dediğin şanslı
olur}'''a kadar gelen düşünceler var. Eğer ekosistemimizde böyle
düşünenler varsa bu öncelikle başarısızlık kültürüne inanmadıklarını
gösterir. Ayrıca bu konu bana hep şu hikayeyi hatırlatır. Bir şirkete
bir satışçı alınacaktır ve belli sayıda uygun aday kalır. Şirket sahibi
sekreterine başvuruların yarısını çöpe atmasını ve kalanlar arasından
birini seçip onu işe almasını ister. Sekreter şaşırır ve madem içinden
bir tanesini tesadüfen seçeceksek neden tamamının içinden seçim
yapmadıklarını sorar. Şirket sahibi şöyler der: ``Şirketimde şanssızlara
yer yok.''

\hypertarget{b-fazla-finansman-konusunda}{%
\subsubsection{b) Fazla finansman
konusunda}\label{b-fazla-finansman-konusunda}}

genişlet ise ``\emph{girişimcilik para olmadan yapılabilecek bir şey
değil}'' şeklinde düşünceler var. Bunları savunanların da ekosisteme
cebinde beş kuruş olmadan giren kişilerin nasıl dünyanın en zengin
insanlarına dönüştüklerinin hikayesini iyi kavramadıklarını düşünüyorum.
Ayrıca bu, şu yazıda bahsettiğim aslında muazzam sayıda insanın
yapabileceği şeylerin belli kısıtlar nedeniyle nasıl çok az sayıdaki
kişi tarafından yapıldığını irdelediğim konuya geliyor. Girişimciliği
sadece finansmanı olan girişimcilerden, finansmanı olmayan girişimcilere
de açmak toplam faydayı şaşırtıcı şekilde arttırdı. Dolayısıyla cebinde
para olmayan kişilerin de girişimcilik yapıp yapamayacağı bir bilinmez
olmaktan çoktan çıkmıştır.

\end{document}
